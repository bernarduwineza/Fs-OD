%==============================================================================
% This is a document for writing the proposal for EE260 project. 
%
% It uses the NeurIPS (formerly NIPS) style. When making changes, please refer 
% to the style file and example given (../nips_style/neurips_2020.tex)

% Please leave comments as you see it necessary; it will help others 
% who also need to modify the document. 
% -JBU
%==============================================================================

\documentclass{article}
% if you need to pass options to natbib, use, e.g.:
%     \PassOptionsToPackage{numbers, compress}{natbib}
% before loading neurips_2020

% ready for submission
% \usepackage{neurips_2020}

% to compile a preprint version, e.g., for submission to arXiv, add add the
% [preprint] option:
%     \usepackage[preprint]{neurips_2020}

% to compile a camera-ready version, add the [final] option, e.g.:
%     \usepackage[final]{neurips_2020}

% to avoid loading the natbib package, add option nonatbib:
     \usepackage[preprint,nonatbib]{./../nips_style/neurips_2020}

\usepackage[utf8]{inputenc} % allow utf-8 input
\usepackage[T1]{fontenc}    % use 8-bit T1 fonts
\usepackage{hyperref}       % hyperlinks
\hypersetup{
    colorlinks=true,
    linkcolor=blue,
    filecolor=magenta,      
    urlcolor=cyan,
}

\usepackage{url}
\urlstyle{same}            % simple URL typesetting
\usepackage{booktabs}       % professional-quality tables
\usepackage{amsfonts}       % blackboard math symbols
\usepackage{nicefrac}       % compact symbols for 1/2, etc.
\usepackage{microtype}      % microtypography

\title{Few Shot Learning for Object Detection: 
Comparison of Fine-tuning and Meta-learning Approaches}

\author{% Please fill in your info
  Jean-Bernard Uwineza 
  %\thanks{Use footnote for providing further information
  %  about author (webpage, alternative address)---\emph{not} for acknowledging
  %  funding agencies.} 
  \\
  %Department of Electrical Engineering \\
  University of California, Riverside\\
  Riverside, CA 92501 \\
  \texttt{buwineza@ee.ucr.edu} \\
  \And
  Chetan Reddy Mudireddy \\
  University of California, Riverside \\
  Riverside, CA 92501 \\
  \texttt{cmudi001@ucr.edu} \\
  \AND
  %
  Om Shankar Ohdar \\
  University of California, Riverside \\
  Riverside, CA 92501 \\
  \texttt{oohda001@ucr.edu} \\
  \And
  %
  Sayak Nag \\
  University of California,Riverside \\
  Riverside, CA 92501 \\
  \texttt{snag005@ucr.edu} \\
  %
  \And
  Vikarn Bhakri \\
  University of California, Riverside \\
  Riverside, CA 92501 \\
  \texttt{vbhak001@ucr.edu} \\
}

\begin{document}

\maketitle


\section{Problem Statement}
A small child visiting a zoo with parents for the first time recognizes a strange animal---a zebra, 
she is told. She keeps walking and a few minutes later, by the infirmary, she recognizes another animal,
only smaller and wobbly this time. ``A baby zebra'', she exclaims. From just this one example, 
she will be able to recognize just about every zebra she will ever see. Not only this, 
she will also be able to make remarkable connections to other animal that are are similar 
to zebras \cite{samuelson2005they}. 

This is an ability machines have yet to acquire. Although machines have surpassed humans in 
visually recognizing objects, they still lack the ability to do so from a few examples. 
Recently,there have been promising advances towards the goal of making machines generalize 
from a few examples via a deep learning method called \textit{few-shot learning}. 
There are many important applications that could benefit from the ability to learn from a few 
samples. Like any other problem, there are various approaches to this this task.
In this project, we propose to evaluate and analyze two of the most promising approaches. 

Few-shot learning has received significant interest in the past few years, 
but mainly for the tasks of classification and rarely for object detection. 
In computer vision, the task of object detection is more challenging since the detector 
not only has to perform recognition of the different kinds of objects present,
it also has to localize them. This is already a challenging task that relies heavily 
on the availability of massive amounts of labeled training data. Now when a new data-point 
is obtained belonging to a novel category, adapting the model becomes a very difficult 
task especially when the new category contains a few samples. Recently, meta learning techniques 
have been proposed for adapting deep models to novel categories. However, they are not easily 
extendable to the task of object detection. Take for example the Matching \cite{VinyalsBLKW16} 
and Prototypical Networks \cite{snell2017prototypical}, building prototypes of objects is 
much more difficult than building prototype of the categories. Another approach that is being 
explored by researchers is to provide ways to fine-tune the detection layers of 
deep models to adapt to the new categories \cite{wang2020frustratingly}.

In this project we aim to do a comparative study of meta-learning and fine-tuning approaches 
towards object detection. We aim to experiment on benchmark datasets such as 
COCO \cite{LinMBHPRDZ14} and PASCAL \cite{Everingham10} and also extend these approaches 
towards 3D object detection with the KITTI dataset \cite{Geiger2013IJRR}.

In addition, we plan on examining how many shots are necessary to reach comparable accuracy relative to 
conventional detection approaches. To this end, we will attempt to develop a metric that assesses
the model's knowledge, and requests additional labeled examples if it has not reached a certain accuracy. This will allow us to further compare the performance of the two approaches. 

\section{Work Plan}

We plan to work on the meta-learning and fine-tuning based approaches simultaneously. 
Sayak and Bernard will focus on meta-learning-based approaches, while Om, Chetan and Vikarn will work 
on fine-tuning approaches. The experimental results will be compiled by Om and Vikarn while Sayak, 
Bernard and Chetan will work on final report compilation. 

























\bibliographystyle{IEEEtran}
\bibliography{References}


\end{document}

