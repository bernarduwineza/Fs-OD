%==============================================================================
% This is a document for writing the proposal for EE260 project. 
%
% It uses the NeurIPS (formerly NIPS) style. When making changes, please refer 
% to the style file and example given (../nips_style/neurips_2020.tex)

% Please leave comments as you see it necessary; it will help others 
% who also need to modify the document. 
% -JBU
%==============================================================================

\documentclass{article}
% if you need to pass options to natbib, use, e.g.:
%     \PassOptionsToPackage{numbers, compress}{natbib}
% before loading neurips_2020

% ready for submission
% \usepackage{neurips_2020}

% to compile a preprint version, e.g., for submission to arXiv, add add the
% [preprint] option:
%     \usepackage[preprint]{neurips_2020}

% to compile a camera-ready version, add the [final] option, e.g.:
%     \usepackage[final]{neurips_2020}

% to avoid loading the natbib package, add option nonatbib:
     \usepackage[preprint]{./../nips_style/neurips_2020}

\usepackage[utf8]{inputenc} % allow utf-8 input
\usepackage[T1]{fontenc}    % use 8-bit T1 fonts
\usepackage{hyperref}       % hyperlinks
\usepackage{url}            % simple URL typesetting
\usepackage{booktabs}       % professional-quality tables
\usepackage{amsfonts}       % blackboard math symbols
\usepackage{nicefrac}       % compact symbols for 1/2, etc.
\usepackage{microtype}      % microtypography

\title{Few Shot Learning for Object Detection: 
Comparison of Fine-tuning and Meta-learning Approaches}

\author{% Please fill in your info
  Jean-Bernard Uwineza 
  %\thanks{Use footnote for providing further information
  %  about author (webpage, alternative address)---\emph{not} for acknowledging
  %  funding agencies.} 
  \\
  %Department of Electrical Engineering \\
  University of California, Riverside\\
  Riverside, CA 92501 \\
  \texttt{buwineza@ee.ucr.edu} \\
  \And
  Chetan Reddy Mudireddy \\
  University of California, Riverside \\
  Riverside, CA 92501 \\
  \texttt{cmudi001@ucr.edu} \\
  \AND
  %
  Om Shankar Ohdar \\
  University of California, Riverside \\
  Riverside, CA 92501 \\
  \texttt{oohda001@ucr.edu} \\
  \And
  %
  Sayak Nag \\
  University of California,Riverside \\
  Riverside, CA 92501 \\
  \texttt{snag005@uvr.edu} \\
  %
  \And
  Vikarn Bhakri \\
  University of California, Riverside \\
  Riverside, CA 92501 \\
  \texttt{vbhak001@ucr.edu} \\
}

\begin{document}

\maketitle

\begin{abstract} % TO BE REMOVED 
  The abstract paragraph should be indented \nicefrac{1}{2}~inch (3~picas) on
  both the left- and right-hand margins. Use 10~point type, with a vertical
  spacing (leading) of 11~points.  The word \textbf{Abstract} must be centered,
  bold, and in point size 12. Two line spaces precede the abstract. The abstract
  must be limited to one paragraph.
\end{abstract}

\section{Proposal}

A small child visiting a zoo with parents for the first time recognizes a strange animal---a zebra, she is told. She keeps walking and a few minutes later, by the infirmary, she recognizes another animal, only smaller and wobbly this time. ``A baby zebra'', she exclaims. From just this one example, she will be able to recognize just about every zebra she will ever see. Not only this, she will also be able to make remarkable connections to other animal that are are similar to zebras. This is an ability machines have yet to acquire. Although machines have surpassed humans in visually recognizing objects, they still lack the ability to do so from a few examples. Recently, there have been promising advances towards the goal of making machines generalize from a few examples via a deep learning method called \textit{few-shot learning}. There are many important applications that could benefit from the ability to learn from a few samples. Like any other problem, there are various approaches to this this task. In this project, we propose to evaluate and analyze two of the most promising approaches. 

Few-shot learning have received significant interest in the past few years, but mainly for the tasks of classification and rarely for object detection. In deep computer vision, the task of object detection is a more challenging one. Not only does the detector have to perform recognition of the different kinds of objects present, it also has to locate their exact locations. This is already a challenging task that relies heavily on the availability of massive amounts of labeled training data. When given base classes with sufficient examples and novel classes with few examples, few-shot learning attempts to find a model that will be able to correctly detect both base and novel objects at test time. 


\end{document}

